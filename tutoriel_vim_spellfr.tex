% Copyright (c) 2008,2014,2015 Jérémie DECOCK (http://www.jdhp.org)

% This document is provided under the terms of the "Creative Commons BY-SA" license.
% For more details, read the "COPYING/legalcode.html" enclosed file or
% the "http://creativecommons.org/licenses/by-sa/4.0/legalcode" web page.

\documentclass{article}

\input{article_packages.tex}
\input{commands.tex}

\sloppy

%%%%%%%%%%%%%%%%%%%%%%%%%%%%%%%%%%%%%%%%%%%%%%%%%%%%%%%%%%%%%%%%%%%%%%%%%%%%%%%

\title{Introduction à la correction orthographique avec Vim}

\author{Jérémie \textsc{Decock} \\ \url{http://www.jdhp.org}}

\date{10 février 2008}

\hypersetup{
	pdftitle={Introduction à la correction orthographique avec Vim}, % title
	pdfauthor={Jérémie DECOCK},                               % author
	pdfsubject={Introduction à la correction orthographique avec Vim}, % subject of the document
	pdfkeywords={vim, correction orthographique, orthographe, dictionnaire, spell} % list of keywords
}

%%%%%%%%%%%%%%%%%%%%%%%%%%%%%%%%%%%%%%%%%%%%%%%%%%

\begin{document}

\maketitle

\section{Préambule}
Ce document explique comment installer et utiliser la correction orthographique
pour la langue française dans l'éditeur Vim. Il a été écrit pour fonctionner
sur n'importe quel système GNU/Linux et devrait fonctionner à l'identique sur
les systèmes BSD (non testé).

À partir de maintenant, nous supposons que Vim est correctement installé et
configuré.\\

La documentation de référence peut être consultée depuis Vim en tapant
\og{}:help\fg{} ou sur le web à l'adresse suivante :
\url{http://vimdoc.sourceforge.net/htmldoc/spell.html}.

%%%%%%%%%%%%%%%%%%%%%%%%%%%%%%%%%%%%%%%%%%%%%%%%%%

\section{Installer les fichiers dictionnaires}
Pour installer le dictionnaire français sur un système GNU/Linux récent, il
faut copier les fichiers suivants dans le répertoire
\textasciitilde{}/.vim/spell/ :
\begin{itemize}
	\item \url{http://ftp.vim.org/vim/runtime/spell/fr.utf-8.spl}
	\item \url{http://ftp.vim.org/vim/runtime/spell/fr.utf-8.sug}
\end{itemize}
~\\
Sur les anciens systèmes GNU/Linux, il est possible que le codage par défaut ne
soit pas l'UTF-8 mais l'ISO 8859-1 ou l'ISO 8859-15. Dans ce cas, il faut
télécharger les fichiers correspondant sur la page
\url{http://ftp.vim.org/vim/runtime/spell/}.

%%%%%%%%%%%%%%%%%%%%%%%%%%%%%%%%%%%%%%%%%%%%%%%%%%

\section{Activer la correction orthographique}
Maintenant que les dictionnaires sont installés, on peut les utiliser dans Vim
en tapant la commande suivante :

\begin{verbatim}
:setlocal spell spelllang=fr
\end{verbatim}

Les fautes d'orthographe apparaissent alors en rouge. Pour éviter de devoir
retaper cette commande à chaque lancement de Vim, il vaut mieux l'inscrire dans
le fichier \textasciitilde{}/.vimrc.

%%%%%%%%%%%%%%%%%%%%%%%%%%%%%%%%%%%%%%%%%%%%%%%%%%

\section{Utiliser le correcteur orthographique}
Voici les principales commandes à connaître :
\begin{itemize}
    \item Pour corriger un mot, il faut placer le curseur dessus et taper
        \og{}z=\fg{}. Une liste de propositions apparaît alors.
    \item Pour corriger toutes les occurrences de ce mot dans le document
        édité, on utilise la commande \og{}:spellr\fg{}.
    \item Pour atteindre la prochaine faute, on tape \og{}]s\fg{}. Pour
        effectuer la même recherche vers le haut, on utilise \og [s \fg.
    \item Pour désactiver et réactiver la correction orthographique, il faut
        utiliser les commandes \og{}:set spell\fg{} et \og{}:set nospell\fg{}.
\end{itemize}

%%%%%%%%%%%%%%%%%%%%%%%%%%%%%%%%%%%%%%%%%%%%%%%%%%

\section{Personnaliser le dictionnaire}
Un dictionnaire personnalisé permet d'enrichir le correcteur en lui indiquant
les mots qu'il ne doit pas considérer comme des fautes (les noms et les prénoms
par exemple).

On peut ajouter un mot en plaçant le curseur dessus et en tapant \og{}zg\fg{}.
Pour annuler cette opération, on utilise \og{}zug\fg{}.

%%%%%%%%%%%%%%%%%%%%%%%%%%%%%%%%%%%%%%%%%%%%%%%%%%

\section{Pour conclure}
La correction orthographique est très efficace dans Vim. Elle a entre autre
l'avantage de tenir compte des spécificités du type de fichier
édité\footnote{À condition d'avoir activé la reconnaissance syntaxique avec
la commande \og{}:syntax~on\fg{}}. Par exemple, quand on active la correction
orthographique sur un fichier .c ou .java, Vim ne va corriger que les
déclarations de chaînes de caractères entre guillemets et les commentaires. De
même, lorsque l'on travaille avec un fichier .tex, il ne va pas tenir compte du
balisage et des formules.

Malheureusement et comme souvent avec les éditeurs libres la grammaire n'est
pas prise en compte.


% Bibliography %%%%%%%%%%%%%%%%%%%%%%%%%%%%%%%%%%%%%%%%%%%%%%%%%%%%%%%%%%%%%%%%

%\nocite{decock:hal-00755663}  % fait apparaitre le document dans la bibliographie sans le citer !
\nocite{*}                    % fait apparaitre TOUS les documents du .bib dans la bibliographie sans les citer !

\bibliographystyle{plain}    % name of the .bst file (bibliography style)
\bibliography{bibliography}  % name of the .bib file (without the file name extension)


% Section License %%%%%%%%%%%%%%%%%%%%%%%%%%%%%%%%%%%%%%%%%%%%%%%%%%%%%%%%%%%%%

\ifpdf
    \vfill % Go to the bottom of the page...
    \begin{center}
        \href{http://creativecommons.org/licenses/by-sa/4.0/}{\includegraphics[width=.15\linewidth]{fig/cc_by_sa_small}}\\
        \small{Creative Commons BY-SA}
    \end{center}
\else
    % HeVeA
    \section*{License}\label{sec:license}

    \begin{rawhtml}

        <div>
            <a rel="license" href="http://creativecommons.org/licenses/by-sa/4.0/">
                <img alt="Licence Creative Commons" style="border-width:0" src="https://i.creativecommons.org/l/by-sa/4.0/80x15.png" />
            </a>
            <br />
            <span xmlns:dct="http://purl.org/dc/terms/" href="http://purl.org/dc/dcmitype/Text" property="dct:title" rel="dct:type">Introduction à la correction orthographique avec Vim</span> de <a xmlns:cc="http://creativecommons.org/ns#" href="http://www.jdhp.org" property="cc:attributionName" rel="cc:attributionURL">Jérémie Decock</a> est mis à disposition selon les termes de la <a rel="license" href="http://creativecommons.org/licenses/by-sa/4.0/">licence Creative Commons Attribution -  Partage dans les Mêmes Conditions 4.0 International</a>.
        </div>

    \end{rawhtml}
\fi

\end{document}
